\documentclass[a4paper, 11pt]{article}
% MLA format
\usepackage[letterpaper]{geometry}
%\usepackage{times}
\geometry{top=1.0in, bottom=1.0in, left=1.0in, right=1.0in}
\usepackage{fancyhdr}
\pagestyle{fancy}
\lhead{} 
\chead{} 
\rhead{Simmons \thepage} 
\lfoot{} 
\cfoot{} 
\rfoot{}
\renewcommand{\headrulewidth}{0pt} 
\renewcommand{\footrulewidth}{0pt} 

%\usepackage{mdwlist}
\usepackage{enumitem}
\setlist{  
  listparindent=\parindent,
  parsep=0pt,
}
\title{Homework 9}
\author{Mark Simmons}
\date{April 24, 2020}


% Typing in IPA
\usepackage{tipa}

% Sentence trees
\usepackage{tikz}
\usepackage{tikz-qtree}
\usepackage{lscape}
\usepackage{graphicx}
\usepackage{pstricks}
\usepackage{tree-dvips}
\tikzset{level distance=30pt,
sibling distance=6pt,
every tree node/.style={align=center},
}

% Glosses
\usepackage{linguex}

\begin{document}

\maketitle



\begin{enumerate}[label=\textbf{\arabic*.}]

\item Technical

\begin{enumerate}[label=(\alph*)]

	% The money was hidden in the drawer.
	\item Deep Structure\\
	\bigskip
	\leavevmode\vadjust{\vspace{-\baselineskip}}\newline
	%\begin{landscape}
	\noindent{\begin{tikzpicture}
	{\small \Tree
	[.CP {}
	[.C' C\\$\emptyset_{[-Q]}$
	[.TP {}
	[.T' T\\-ed
		[.VP 
        [.V' V\\be
            [.V'
                [.V' V\\hidden
                    [.DP \edge[roof]; {The money} ]
                ]%V'
                [.PP \edge[roof]; {in the drawer.} ]
            ]%V'
		]%V'
		{} ]%VP
	]%T'
	]%TP
	]%C'
	]%CP
	}
	\end{tikzpicture}}

	\pagebreak

	\noindent Surface Structure\\
	\bigskip
	\leavevmode\vadjust{\vspace{-\baselineskip}}\newline
	%\begin{landscape}
	\noindent{\begin{tikzpicture}
	{\small \Tree
	[.CP [.DP \edge[roof]; \node(subSpecT){The money$_{i}$}; ]
	[.C' C\\$\emptyset_{[-Q]}$
	[.TP {}
	[.T' \node(vT){T\\be-ed$_{j}$};
		[.VP {}
        [.V' \node(traceV){V\\t$_{j}$};
            [.V'
                [.V' V\\hidden
                    \node(traceSub){t$_{i}$};
                ]%V'
                [.PP \edge[roof]; {in the drawer.} ]
            ]%V'
		]%V'
		]%VP
	]%T'
	]%TP
	]%C'
	]%CP
	}
    \draw[dashed,->] (traceV)..controls +(south:1) and +(south:4)..(vT);
	\draw[semithick,->] (traceSub)..controls +(south:3) and +(south:6)..(subSpecT);
    \end{tikzpicture}}

	\pagebreak

	% It seems that Sonny loves Cher.
	\item Deep Structure\\
	\bigskip
	\leavevmode\vadjust{\vspace{-\baselineskip}}\newline
	%\begin{landscape}
	\noindent{\begin{tikzpicture}
	{\small \Tree
	[.CP {}
	[.C' C\\$\emptyset_{[-Q]}$
	[.TP {}
	[.T' T\\-s
		[.VP {}
        [.V' V\\seem
			[.CP {}
			[.C' C\\that
				[.TP {}
				[.T' T\\-s
				[.VP [.DP \edge[roof]; {Sonny} ]
				[.V' V\\love [.DP \edge[roof]; {Cher.} ]
				]%V'
				]%VP
				]%T'
				]%TP
			]%C'
            ]%CP
		]%V'
		]%VP
	]%T'
	]%TP
	]%C'
	]%CP
	}
	\end{tikzpicture}}

	\pagebreak

	\noindent Surface Structure\\
	\bigskip
	\leavevmode\vadjust{\vspace{-\baselineskip}}\newline
	%\begin{landscape}
	\noindent{\begin{tikzpicture}
	{\small \Tree
	[.CP {}
	[.C' C\\$\emptyset_{[-Q]}$
	[.TP \node(subSpecT)[draw,circle]{DP\\It$_{i}$};
	[.T' \node(tTrace){};
		[.VP \node(traceSub){t$_{i}$};
        [.V' \node(v){V\\seem-s};
			[.CP {}
			[.C' C\\that
				[.TP [.DP \edge[roof]; \node(subSpecTpLower){Sonny$_{j}$}; ]
				[.T' \node(tTraceLower){};
				[.VP \node(traceSubLower){t$_{j}$};
				[.V' \node(vLower){V\\love-s}; [.DP \edge[roof]; {Cher.} ]
				]%V'
				]%VP
				]%T'
				]%TP
			]%C'
            ]%CP
		]%V'
		]%VP
	]%T'
	]%TP
	]%C'
	]%CP
	}
	\draw[dashed,->] (tTraceLower)..controls +(south:3) and +(south:4)..(vLower);
    \draw[dashed,->] (tTrace)..controls +(south:3) and +(south:4)..(v);
	\draw[semithick,->] (traceSub)..controls +(south:1) and +(south:3)..(subSpecT);
	\draw[semithick,->] (traceSubLower)..controls +(south:1) and +(south:3)..(subSpecTpLower);

    \end{tikzpicture}}

	\pagebreak

	% Donny appears to have been kissed by the puppy.
	\item Deep Structure\\
	\tikzset{level distance=15pt,
    sibling distance=3pt,
    }
	\bigskip
	\leavevmode\vadjust{\vspace{-\baselineskip}}\newline
	%\begin{landscape}
	\noindent{\begin{tikzpicture}
	{\tiny \Tree
	[.CP {}
	[.C' C\\$\emptyset_{[-Q]}$
	[.TP {}
	[.T' T\\-s
	[.VP {}
	[.V' \node(v){V\\appear};
		[.CP {}
		[.C' C\\$\emptyset_{[-Q]}$
			[.TP {}
			[.T' T\\to
			[.VP {}
			[.V' V\\have
				[.VP {} [.V' V\\been
					[.VP {} 
					[.V'
						[.V' V\\kissed [.DP \edge[roof]; {Donny} ] ]
						[.PP \edge[roof]; {by the puppy.} ]
					]%V'
					]%VP
				] ]%V', VP
			]%V'
			]%VP
			]%T'
			]%TP
		]%C'
		]%CP
	]%V'
	]%VP
	]%T'
	]%TP
	]%C'
	]%CP
	}
	\end{tikzpicture}}

	\pagebreak

	\noindent Surface Structure\\
	\bigskip
	\leavevmode\vadjust{\vspace{-\baselineskip}}\newline
	%\begin{landscape}
	\noindent{\begin{tikzpicture}
	{\tiny \Tree
	[.CP {}
	[.C' C\\$\emptyset_{[-Q]}$
	[.TP [.DP \edge[roof]; \node(specTP){Donny$_{i}$}; ]
	[.T' \node(traceT){T};
	[.VP {}
	[.V' \node(v){V\\appear-s};
		[.CP {}
		[.C' C\\$\emptyset_{[-Q]}$
			[.TP \node(specToKissTrace){t$_{i}$};
			[.T' T\\to
			[.VP {}
			[.V' V\\have
				[.VP {} [.V' V\\been
					[.VP {} 
					[.V'
						[.V' V\\kissed \node(objTrace){t$_{i}$}; ]
						[.PP \edge[roof]; {by the puppy.} ]
					]%V'
					]%VP
				] ]%V', VP
			]%V'
			]%VP
			]%T'
			]%TP
		]%C'
		]%CP
	]%V'
	]%VP
	]%T'
	]%TP
	]%C'
	]%CP
	}
	\draw[dashed,->] (traceT)..controls +(south:2) and +(south:2)..(v);
	\draw[semithick,->] (objTrace)..controls +(south:4) and +(south:4)..(specToKissTrace);
	\draw[semithick,->] (specToKissTrace)..controls +(south:2) and +(south:6)..(specTP);


	\end{tikzpicture}}

	\pagebreak

	% John is likely to appear to have been defeated.
	\item Deep Structure\\
	\tikzset{level distance=15pt,
    sibling distance=3pt,
    }
	\bigskip
	\leavevmode\vadjust{\vspace{-\baselineskip}}\newline
	%\begin{landscape}
	\noindent{\begin{tikzpicture}
	{\tiny \Tree
	[.CP {}
	[.C' C\\$\emptyset_{[-Q]}$
	[.TP {}
	[.T' T\\-s
	[.VP {}
	[.V' V\\be
	[.AdjP {}
	[.Adj' Adj\\likely
		[.CP {}
		[.C' C\\$\emptyset_{[-Q]}$
		[.TP {}
		[.T' T\\to
		[.VP {}
		[.V' V\\appear
			[.CP {}
			[.C' C\\$\emptyset_{[-Q]}$
				[.TP {}
				[.T' T\\to
				[.VP {}
				[.V' V\\have
					[.VP {} [.V' V\\been
						[.VP {} [.V' V\\defeated [.DP \edge[roof]; {John} ] ] ]%DP, V', VP
					] ]%V', VP
				]%V'
				]%VP
				]%T'
				]%TP
			]%C'
			]%CP
		]%V'
		]%VP
		]%T'
		]%TP
		]%C'
		]%CP
	]%AdjP
	]%Adj'
	]%VP
	]%V'
	]%T'
	]%TP
	]%C'
	]%CP
	}
	\end{tikzpicture}}

	\pagebreak

	\noindent Surface Structure\\
	\bigskip
	\leavevmode\vadjust{\vspace{-\baselineskip}}\newline
	%\begin{landscape}
	\noindent{\begin{tikzpicture}
	{\tiny \Tree
	[.CP {}
	[.C' C\\$\emptyset_{[-Q]}$
	[.TP [.DP \edge[roof]; \node(specTP){John$_{i}$}; ]
	[.T' \node(vT){T\\be-s$_{j}$};
	[.VP {}
	[.V' \node(vTrace){t$_{j}$};
	[.AdjP {}
	[.Adj' Adj\\likely
		[.CP {}
		[.C' C\\$\emptyset_{[-Q]}$
		[.TP \node(specToAppearTrace){t$_{i}$};
		[.T' T\\to
		[.VP {}
		[.V' V\\appear
			[.CP {}
			[.C' C\\$\emptyset_{[-Q]}$
				[.TP \node(specToDefeatTrace){t$_{i}$};
				[.T' T\\to
				[.VP {}
				[.VP {} [.V' V\\have
					[.VP {} [.V' V\\been
						[.VP {} [.V' V\\defeated \node(objTrace){t$_{i}$}; ] ]%V', VP
					] ]%V', VP
				] ]%V', VP
				]%VP
				]%T'
				]%TP
			]%C'
			]%CP
		]%V'
		]%VP
		]%T'
		]%TP
		]%C'
		]%CP
	]%AdjP
	]%Adj'
	]%VP
	]%V'
	]%T'
	]%TP
	]%C'
	]%CP
	}
	\draw[dashed,->] (vTrace)..controls +(south:2) and +(south:2)..(vT);
	\draw[semithick,->] (objTrace)..controls +(south:4) and +(south:3)..(specToDefeatTrace);
	\draw[semithick,->] (specToDefeatTrace)..controls +(south:4) and +(south:4)..(specToAppearTrace);
	\draw[semithick,->] (specToAppearTrace)..controls +(south:2) and +(south:6)..(specTP);


	\end{tikzpicture}}

	
    \end{enumerate}

\item \textbf{Argumentation}

	In English passive sentences, the semantic patient of the verb is manifested syntactically and morphologically
	as the subject, as in (1).

	\ex.
	  \a.
		We elected him.
	  \b.
	  	He was elected.


	The object of \emph{elect} in (1a) behaves as a canonical object in English; it follows the verb and is
	marked with accusative case. In (1b), however, the semantic object has been fronted to before the verb,
	where the subject normally resides, and is marked with nominative case. The Icelandic passive in (2)
	behaves similarly.

	\ex.
	\ag.
	  Vi\textipa{D} kusum stelpuna.\\
	  1\sc{pl.nom} elected girl\sc{.def.sg.f.acc}\\
	  \trans `We elected the girl.'
	\bg.
	  Stelpan var kosin.\\
	  girl\sc{.def.sg.f.nom} was elected\\
	  \trans `The girl was elected.'

	In (2a) \emph{stelpuna} follows the verb and is marked with accusative case, while in (2b) it precedes the verb
	and is marked with nominative case. Thus, it seems that in general the semantic object of a passive verb
	behaves syntactically and morphologically like a subject.

	However, an issue arises with verbs that take so-called ``quirky case'' DPs as direct objects, rather than accusative DPs.

	\ex.
	\ag.
	  Vi\textipa{D} hjálpu\textipa{D}um/björgu\textipa{D}um/heilsu\textipa{D}um stelpunum.\\
	  1\sc{pl.nom} helped/rescued/greeted girl\sc{.def.sg.f.dat}\\
	  \trans `We helped/rescued/greeted the girl.'
	\bg.
	  Vi\textipa{D} söknu\textipa{D}um/leitu\textipa{D}um/g{\ae}ttum hennar.\\
	  1\sc{pl.nom} missed/searched\_for/looked\_after\ 3sc{.sg.f.gen}\\
	  \trans `We missed/searched for/looked after her.'

	Our current theory of nominal case predicts that the use of dative in (3a) and genitive and (3b) are mere morphological
	proxies for an abstract [+uACC] Case marked syntactically on the DP. When the verb is passivized, it loses its
	ability to check for accusative case, requiring the object to move to subject position, which checks for [+uNOM].
	If the dative and genitive observed above are indeed proxies for [+uACC], we would expect the DPs to lose the
	oblique case inflection and manifest as morphologically nominative when moved to subject position in a passive sentence.
	As the following data show, that is not the case (ba dum tsss).

	\ex.
	\ag.
	  Henni var hjálpa\textipa{D}/bjaraga\textipa{D}/heilsa\textipa{D}.\\
	  3\sc{.sg.f.dat} was helped/rescued/greeted\\
	  \trans `She was helped/rescued/greeted.'
	\bg.
	  Hennar var sakna\textipa{D}/leita\textipa{D}/g{\ae}tt.\\
	  3\sc{.def.sg.f.gen} was missed/searched\_for/looked\_after\\
	  \trans `She was missed/searched for/looked after.'

	As (4) demonstrates, these nouns maintain quirky case even when subject of a passive sentence. Since, as stated above
	quirky case is assumed to be a proxy for an abstract [+uACC] feature, it appears that these sentences have an accusative
	DP in subject position. This is a problem for our theory, which predicts that subject position checks only for [+uNOM],
	and should not allow nouns marked for any other case in this posititon.

	
	However, there is a possible explanation other than assuming that \emph{henni} and \emph{hennar} are underlyingly
	Accusative in (4). It is possible that they have accusative Case in (3) and nominative case in (4), but surface as dative and genitive because
	of a post-syntactic rule that overrides whatever morphology is expected from abstract Case.
	
	Before we imagine what such a rule could be, let us consider the behavior of phrasal verbs in English in passive sentences.

	\ex.
	\a.
	  We looked after her.
	\b.
	  She was looked after.
	
	In (5a), the direct object \emph{her} is expressed as the object of the preposition \emph{after}. However, this is not
	what the compositional semantics entail, otherwise (5a) would by synonymous with \emph{We looked beyond her}. Instead, the
	meaning is closer to \emph{We protected her}. In other words, the direct object of the verb in (5a) is expressed morphologically
	as if it were an oblique argument. This is similar to Icelandic's quirky case. Where English quirky verbs have an idiosyncratic
	preposition that governs the direct object, Icelandic quirky verbs assign an idiosyncratic case affix to their direct object.
	I write these affixes as $\emptyset_{[+DAT]}$ and $\emptyset_{[+GEN]}$.

	The key here is the distinction between a preposition (that is, a lexeme) and an affix. A lexeme can stand on its own in SS,
	for this reason appears separate from the patient DP in (5b). Trying to front it along with the pronoun results in an ungrammatical
	sentence.

	\ex. *After her was looked.

	An affix, however, must have a lexical host. Thus, the affix has no choice but to raise with its nominal host in order to
	satisfy the Stranded Affix Filter.

	Other forms of the passive require further explanation.

	\ex.
	\ag.
		B\ae kurnar voru lesnar.\\
		book\sc{.pl.nom.def} were returned\\
		\trans `The books were returned'
	\bg.
		\textthorn a\textipa{D} var lesnar fjórar b\ae kur.\\
		\sc{expl}  were read four\sc{.pl.nom} book\sc{.pl.nom}\\
		\trans `Four books were read.'

	Here we see that, in an alternate passive contruction, the patient DP is left in its original position following the
	verb, rather than raised to subject position, and an expletive pronoun occupies the sentence's subject position. As
	expected, the patient DP is not marked accusative, as the passive verb has lost its [+ACC] feature. There is still a problem,
	nonetheless: the patient DP should not be able to stay within the VP at all; our theory predicts that passive verbs should
	not be able to assign any case, which is why the noun phrase is expected to raise, so that it can check its case feature
	against the TP.

	One possible solution to this is to modify the rule for feature-checking on Icelandic nouns. When a noun has no place in
	the sentence to check its case feature, it is realized in SS with the most unmarked morphological case, which in Icelandic
	appears to be the nominative. I will refer to this rule as the Unmarked Nominative Filter. Note how this contrasts with
	English, which (according to the model we have developed so far for it) does not allow for nouns to lack a case feature
	in DS.

	The assignation of nominative morphology to \emph{fjórar b\ae kur} thus occurs post-syntactically, as a ``last resort''
	option since no other case has been assigned to it. Thus, we might expect that other post-syntactic processes might
	assign case before the Unmarked Nominative Filter occurs - for example, if a noun is assigned a case by a null affix
	that governs it morphologically.

	\ex.
	\ag.
		Bókunum var skila\textipa{D}.\\
		book\sc{.pl.dat.def} was returned\\
		\trans `The books were returned'
	\bg.
		\textthorn a\textipa{D} var skila\textipa{D} fjórum bókum.\\
		\sc{expl}  was returned four\sc{.pl.dat} book\sc{.pl.dat}\\
		\trans `Four books were returned.'

	Since the verb \emph{skila} has a quirky-case affix that associates onto the object noun and gives it dative morphology,
	we see it marked so in the SS, even though underlyingly the object DP lacks abstract Case.

	Another type of passive appears to disprove the posited Unmarked Nominative Filter.

	\ex.
	\ag.
		\textthorn a\textipa{D} var bari\textipa{D} mig.\\
		\sc{expl} was beaten 1\sc{.sg.acc}\\
		\trans `I was beaten.'

	This sentence is identical in structure to the passive in (7b), save that the patient DP is marked accusative rather than
	nominative. This violates the proposed Unmarked Nominative Filter. However, the reason that the nominative case was used
	in the rule was not for any underlying syntactic motivation, but rather simply because it seemed to be the most unmarked
	case in Icelandic. Given the recency of this new construction, I propose that it could be motivated by a shift in 
	markedness, namely, that the accusative case in modern Icelandic is becoming less marked than the nominative. Compare
	sentences in English such as \emph{Me and my friends went to the store} - though \emph{me} is part of the subject DP and,
	thus, should be nominative, it appears accusative, seemingly because it is separate from the main verb by a conjunction.
	The fact that the accusative case is assigned to this pronoun when there is no syntactic or semantic motivation to do so
	suggests that the accusative is less marked overall than nominative in English. Thus, the same could be true of modern
	Icelandic, which would reconcile the data above with the theory established in this paper.

	Thus, I propose that the differences in morphology and word order observed in English and Icelandic passives can be
	explained via parametric variations between the language. First, Icelandic verbs that take a quirky case object
	instead of accusative assign this case via a lexicalized affix that functions similar to the particle in English verbs
	such as \emph{look after}, which then follow the object through any movement or transformations due to the Stranded Affix
	Filter. Further, where English does not allow the object of a passivized
	verb to remain in-situ as complement to V and be featureless with regards to case in DS, Icelandic does allow so when an expletive subject
	occupies specifier of TP position. I finally argue that Icelandic employs a filter rule which assigns any such stranded
	DPs the most unmarked morphological case if no other rule assigns them a case first.

\item \textbf{Argumentation}\\
	Several English verbs can occur with an embedded clause. In X-Bar theory, we say that these verbs
	take the embedded clause as their complement.

	\ex. John believes that Bill lied.

	Here, ``that Bill lied'' is the complement of the verb \emph{believe}. The complement clause here is a
	tensed clause introduced by the \emph{that}. However, \emph{believe} can also take a non-tensed clause
	as its complement, such as an infinitive clause.
	
	\ex. John believes Bill to have lied.

	Though \emph{believe} here appears to possess the same argument structure as it does in (10), I will argue
	that \emph{Bill} actually moves from being subject of the embedded clause in DS to being object of the
	verb \emph{believe} in the matrix clause in SS, meaning that \emph{believe} in (2) has two separate
	complements.

	Disregarding theory, there is reason to believe \emph{Bill} is the semantic subject of the verb \emph{lie},
	is syntactically the object of \emph{believe}. If we replace \emph{Bill} in (11) with a pronoun, we can see
	it is marked accusative morphologically.

	\ex. 
		\a. John believes him to have lied.
		\b. *John believes he to have lied.

	Furthermore, \emph{believe} can occur with an object noun and no complement clause:

	\ex.
		\a. I believe him.
		\b. John will never believe me.

	Thus, since \emph{believe} can clearly take an accusative object, it is preferable to consider that \emph{Bill}
	in (11) and \emph{him} in (12a) are the syntactic object of \emph{believe}, rather than being part of the
	infinitive clause.

	This fits our understanding of Case Filter as well. A non-finite clause cannot check nominative case on a
	DP, and thus it cannot take a subject. This causes the underlying subject of \emph{lie} to move to the
	nearest case-checking position, which is the object position of the verb in the matrix clause.

	Another theoretical ``probe'' we can apply here is Binding Theory. Since the binding domain for elements
	such as pronouns and anaphors is the lexeme's immediate clause, we can discern whether this clause
	is the matrix Tensed clause, or the embedded infinitive clause.

	As demonstrated below, pronouns in this position cannot be co-indexed with the subject of the matrix
	clause.

	\ex. 
		\a. John$_{i}$ believes him$_{j}$ to have been defeated.
		\b. *John$_{i}$ believes him$_{i}$ to have been defeated.

	Binding Principle B states that pronouns must be free within their binding domain, which is the
	immediate clause. (14b) shows that \emph{him} cannot be coindexed with \emph{John}, though it is
	perfectly grammatical if it indexes any other referent. This suggests that \emph{him} is inside the same
	clause as \emph{John}, and that (14b) is unacceptable because \emph{John} would then bind \emph{him},
	violating Binding Principle B.

	On the other hand, anaphors are capable of being coindexed with \emph{John} in this sentence.

	\ex.
		\a. John$_{i}$ believes himself$_{i}$ to have been defeated.
		\b. John$_{i}$ believes Bob$_{j}$ to have defeated himself$_{j}$.

	Binding Principle A states that anaphors must be bound within their binding domain.
	(14a) proves that \emph{himself} is inside the matrix clause, else it would not be bound by \emph{John}
	and the sentence would be ungrammatical. (15b) raises a bit of an issue - how can \emph{Bob} bind
	\emph{himself} if \emph{Bob} is in a higher clause? It is possible that in (15a) \emph{Bob} binds
	\emph{himself} in DS but not SS, whereas in (15a) \emph{John} binds \emph{himself} in SS but not in DS.

	In fact, this could actually be proof that the subject of the embedded clause is undergoing movement.
	Compare (15a,b) to their follow equivalent sentences.

	\ex.
	\a. *John$_{i}$ believes that himself$_{i}$ has been defeated.
	\b. John$_{i}$ believes that Bob$_{j}$ has defeated himself$_{j}$.

	When \emph{believe} takes a finite clause complement introduced by \emph{that}, the subject of the embedded
	clause remains in the embedded clause; otherwise it would not occur following \emph{that}. Since there is no
	movement in (16), \emph{John} cannot bind \emph{himself} either in DS or SS. However, in (15a), \emph{John} does
	bind \emph{himself} in SS precisely because it has moved into the matrix clause.

	Thus, I propose that when verbs like \emph{believe} take a non-finite clause complement, the subject of the
	embedded clause raises to the matrix clause, manifesting as the object of \emph{believe}. I argue this from
	the observed morphology - the case of the noun in question, and by applying the theoretical rules of Case Filter
	and Binding Theory.

\end{enumerate}

\end{document}