\documentclass[a4paper, 11pt]{article}
% MLA format
\usepackage[letterpaper]{geometry}
%\usepackage{times}
\geometry{top=1.0in, bottom=1.0in, left=1.0in, right=1.0in}
\usepackage{fancyhdr}
\pagestyle{fancy}
\lhead{} 
\chead{} 
\rhead{Simmons \thepage} 
\lfoot{} 
\cfoot{} 
\rfoot{}
\renewcommand{\headrulewidth}{0pt} 
\renewcommand{\footrulewidth}{0pt} 

%\usepackage{mdwlist}
\usepackage{enumitem}
\setlist{  
    listparindent=\parindent,
    parsep=0pt,
}
\title{Final}
\author{Mark Simmons}
\date{May 15, 2020}


% Typing in IPA
\usepackage{tipa}

% Sentence trees
\usepackage{tikz}
\usepackage{tikz-qtree}
\usepackage{lscape}
\usepackage{graphicx}
\usepackage{pstricks}
\usepackage{tree-dvips}
\tikzset{level distance=15pt,
sibling distance=3pt,
every tree node/.style={align=center},
}

% Glosses
\usepackage{linguex}

\begin{document}

\maketitle


\begin{enumerate}[label=\textbf{\arabic*.}]

\item \textbf{Technical}

    \begin{enumerate}[label=(\alph*)]
    \item
    \begin{enumerate}[label=\roman*.]
    \item
    % Which Lego toys was Ezra told to play with?
    \noindent Deep Structure\\
    \bigskip
    \leavevmode\vadjust{\vspace{-\baselineskip}}\newline
    %\begin{landscape}
    \noindent{\begin{tikzpicture}
        {\tiny \Tree
        [.CP {}
        [.C' -$\emptyset_{[+Q +wh]}$
        [.TP {}
        [.T' T\\-ed
        [.VP {} % be told
        [.V' V\\be
          [.VP {} % told
          [.V' V\\told [.DP$_{j}$ \edge[roof]; \node(specTP){Ezra}; ]
            [.CP {}
                [.C' C\\$\emptyset_{[-Q]}$
                    [.TP {}
                    [.T' T\\to
                        [.VP [.DP$_{j}$ \edge[roof]; \node(specToPlay){PRO}; ]
                        [.V' V\\play
                        [.PP {}
                        [.P' P\\with [.DP$_{i}$ \edge[roof]; \node(specCP){which Lego toys}; ]
                        ]%P'
                        ]%PP
                        ]%V'
                        ]%VP
                    ]%T'
                    ]%TP
                ]%C;
            ]%CP
          ]%V'
          ]%VP
        ]%V'
        ]%VP
        ]%T'
        ]%TP
        ]%C'
        ]%CP
        }
    \end{tikzpicture}}

    \pagebreak

    % Which Lego toys was Ezra told to play with?
    \item \noindent Surface Structure\\
    \bigskip
    \leavevmode\vadjust{\vspace{-\baselineskip}}\newline
    %\begin{landscape}
    \noindent{\begin{tikzpicture}
    {\tiny \Tree
    [.CP [.DP$_{i}$ \edge[roof]; \node(specCP){Which Lego toys}; ]
    [.C' \node(vC){C\\be-ed-$\emptyset_{[+Q]}$};
    [.TP [.DP$_{j}$ \edge[roof]; \node(specTP){Ezra}; ]
    [.T' \node(traceT){T};
    [.VP {} % be told
    [.V' \node(specBeTold){t};
      [.VP {} % told
      [.V' V\\told \node(objTrace){t$_{j}$};
        [.CP \node(lowerSpecCP){t$_{i}$};
            [.C' C\\$\emptyset_{[-Q]}$
                [.TP [.DP \edge[roof]; \node(specToPlay){PRO$_{j}$}; ]
                [.T' T\\to
                    [.VP \node(specPlay){t$_{j}$}; % play
                    [.V' V\\play
                    [.PP {}
                    [.P' P\\with \node(whTrace){t$_{i}$};
                    ]%P'
                    ]%PP
                    ]%V'
                    ]%VP
                ]%T'
                ]%TP
            ]%C;
        ]%CP
      ]%V'
      ]%VP
    ]%V'
    ]%VP
    ]%T'
    ]%TP
    ]%C'
    ]%CP
    }
    \draw[dashed,->] (traceT)..controls +(south:3) and +(south:3)..(vC);
    \draw[dashed,->] (specBeTold)..controls +(south:2) and +(south:1)..(traceT);
    \draw[semithick,->] (objTrace)..controls +(south:2) and +(south:2)..(specTP);
    \draw[semithick,->] (specPlay)..controls +(south:2) and +(south:2)..(specToPlay);
    \draw[dotted,->] (whTrace)..controls +(south:5) and +(south:5)..(lowerSpecCP);
    \draw[dotted,->] (lowerSpecCP)..controls +(south:5) and +(south:5)..(specCP);
    \end{tikzpicture}}

    \item Theta Bois\\
    \emph{tell-en ($\rightarrow$ told)}\\
    \begin{tabular}{ |c|c|c| } 
        \hline
        \underline{agent} & patient & proposition \\ 
        DP & DP & CP  \\ 
        \hline
        -en & i & j \\ 
        \hline
    \end{tabular}
    \vspace{5mm}\\
    \emph{play}\\
    \begin{tabular}{ |c|c| } 
        \hline
        \underline{agent} & theme \\ 
        DP & PP  \\ 
        \hline
        k & l \\ 
        \hline
    \end{tabular}
    \vspace{5mm}\\
    Which Lego toys was [Ezra]$_{i}$ told [PRO$_{k}$ to play [with \emph{t}]$_{l}$]$_{j}$?

    \item Case Filter
    \begin{itemize}
        \item \emph{Ezra} This DP is assigned [+\emph{u}NOM] by the verb \emph{told}. Since the
        passive verb told cannot check [+\emph{u}NOM], it moves to spec of TP position to check
        this case.
        \item \emph{which Lego toys} the DP is assigned [+\emph{u}ACC] by the preposition
        \emph{with}. When Case Filter is applied, this DP is still in the PP, so the preposition
        \emph{with} checks its case feature before undergoing \emph{wh}-movement.
    \end{itemize}

    \item Stranded Affix Filter
    \begin{itemize}
        \item \emph{-ed} This affix attaches to the verb \emph{be} when \emph{be} raises to T position.
        \item \emph{$\emptyset_{[+Q]}$} This affix attaches to the verb \emph{be} when \emph{be} raises
        to C position (after picking up \emph{-ed} at T).
    \end{itemize} 

    \item EPP
    \begin{itemize}
        \item In the matrix clause (\emph{Which Lego toys was Ezra told}), the object of the verb
        \emph{told}, \emph{Ezra}, raises to spec-TP position to satisfy the EPP.
        \item In the embedded clause (\emph{PRO to play with t}), the subject of the verb play,
        PRO, raises to spec-TP position to satisfy the EPP.
    \end{itemize} 

    \item MLC\\
    The wh-phrase \emph{which Lego toys} raises to the nearest landing spot, which is
    spec-CP of its original clause (with lexical verb
    \emph{play}) in order to satisfy the MLC, then it raises again to the spec-CP
    position of the 
    clause above it (with lexical verb \emph{told}), where it appears in SS.

    \end{enumerate}

    \pagebreak

    \item
    % Morrissey seemed to wonder when it rained.
    \noindent Deep Structure\\
    \bigskip
    \leavevmode\vadjust{\vspace{-\baselineskip}}\newline
    %\begin{landscape}
    \noindent{\begin{tikzpicture}
    {\tiny \Tree
    [.CP {}
    [.C' C\\$\emptyset_{[-Q]}$
    [.TP {}
    [.T' T\\-ed
    [.VP {} % seemed
    [.V' \node(vSeem){V\\seem};
        [.CP {}
        [.C' C\\$\emptyset_{[-Q]}$
        [.TP {}
        [.T' T\\to
        [.VP [.DP \edge[roof]; \node(specTP){Morrissey}; ]
        [.V' V\\wonder
            [.CP {}
                [.C' C\\$\emptyset_{[-Q +wh]}$
                    [.TP {}
                    [.T' T\\-ed
                        [.VP {} % rain
                        [.V'
                            [.V' \node(vRain){rain}; {} ]
                            [.AdvP \edge[roof]; \node(specCP){when}; ]
                        ]%V'
                        ]%VP
                    ]%T'
                    ]%TP
                ]%C;
            ]%CP
        ]%V'
        ]%VP
        ]%T'
        ]%TP
        ]%C'
        ]%CP
    ]%V'
    ]%VP
    ]%T'
    ]%TP
    ]%C'
    ]%CP
    }
    \end{tikzpicture}}

    \pagebreak

    % Morrissey seemed to wonder when it rained.
    \noindent Surface Structure\\
    \bigskip
    \leavevmode\vadjust{\vspace{-\baselineskip}}\newline
    %\begin{landscape}
    \noindent{\begin{tikzpicture}
    {\tiny \Tree
    [.CP {}
    [.C' C\\$\emptyset_{[-Q]}$
    [.TP [.DP$_{i}$ \edge[roof]; \node(specTP){Morrissey}; ]
    [.T' \node(traceT){T};
    [.VP {} % seemed
    [.V' \node(vSeem){V\\seem-ed};
        [.CP {}
        [.C' C\\$\emptyset_{[-Q]}$
        [.TP \node(specToWonder){t$_{i}$};
        [.T' T\\to
        [.VP \node(specWonder){t$_{i}$};
        [.V' V\\wonder
            [.CP [.AdvP \edge[roof]; \node(specCP){when}; ]
                [.C' $\emptyset_{[-Q, +wh]}$
                    [.TP \fbox{it}
                    [.T' \node(tRained){t};
                        [.VP {} % rain
                        [.V'
                            [.V' \node(vRain){V\\rain-ed}; {} ]
                            \node(whTrace){t};
                        ]%V'
                        ]%VP
                    ]%T'
                    ]%TP
                ]%C;
            ]%CP
        ]%V'
        ]%VP
        ]%T'
        ]%TP
        ]%C'
        ]%CP
    ]%V'
    ]%VP
    ]%T'
    ]%TP
    ]%C'
    ]%CP
    }
    \draw[dashed,->] (traceT)..controls +(south:2) and +(south:2)..(vSeem);
    \draw[dashed,->] (tRained)..controls +(south:2) and +(south:1)..(vRain);
    \draw[dotted,->] (whTrace)..controls +(south:5) and +(south:5)..(specCP);
    \draw[semithick,->] (specToWonder)..controls +(south:2) and +(south:4)..(specTP);
    \draw[semithick,->] (specWonder)..controls +(south:2) and +(south:2)..(specToWonder);
    \end{tikzpicture}}

    \pagebreak

    \item Theta Bois\\
    \emph{wonder}\\
    \begin{tabular}{ |c|c|c| } 
        \hline
        \underline{agent}  & proposition \\ 
        DP & CP  \\ 
        \hline
        i & j \\ 
        \hline
    \end{tabular}
    \vspace{5mm}\\

    [Morrissey]$_{i}$ seemed to t$_{i}$ wonder [when it would rain.]$_{j}$

    \item Case Filter
    \begin{itemize}
        \item \emph{Morrissey} This DP is assigned [+\emph{u}NOM] by the verb \emph{wonder}. Since
        the infinitive verb \emph{to wonder} cannot check for Nominative case, the DP moves to spec of
        TP in the above clause, which, being a tensed clause, can check for Nominative.
    \end{itemize}

    \item Stranded Affix Filter
    \begin{itemize}
        \item \emph{-ed} This affix lowers to the verb \emph{seem}.
    \end{itemize} 

    \item EPP
    \begin{itemize}
        \item In the matrix clause (\emph{Morrissey seemed}), the EPP is satisfied by the DP
        \emph{Morrissey}, which 
        \item In the embedded clause (\emph{to wonder}), the EPP is satisfied by the trace
        \emph{Morrissey} left in spec-TP before raising to the matrix clause.
        \item In the lowermost embedded clause (\emph{when it rained}), the EPP is satisfied by the
        expletive pronoun \emph{it} which is inserted into spec-TP position at SS.
    \end{itemize} 

    \item MLC\\
    The wh-phrase \emph{when} moves to the nearest landing spot, which is
    spec-CP position of its local clause, and remains there at SS,
    satisfying the MLC.

    \end{enumerate}

\item \textbf{Argumentation}

In English, \emph{wh}-pronouns are used to create both content questions and relative clauses.
The pronoun in content questions undergoes \emph{wh}-movement, which raises it to spec-CP position
for whatever CP is marked with the [+wh].

\ex.
    \a. Joe saw him.
    \b. Who did Joe see?
    \c. The man who Joe saw is over there.

In (1b), the accusative pronoun corresponding to \emph{him} in (1a) has been moved to spec-CP,
and for this reason it appears before the subject \emph{Joe}. In (1c) the same pronoun appears
again in spec-CP position, but this time within an embedded relative clause.
I propose that relative pronouns in English undergo \emph{wh}-movement of the same sort seen
in content questions, triggered by the [+wh, -Q] features marked on the relative clause. To argue
this, I will show that relative clauses are subject to the same ``Island Constraints'' as content
questions.

As observed above, a relative pronoun can raise to spec-CP from a lower place within the clause.
It can also raise from an embedded clause to spec-CP of the clause above, as it does in (2b).

\ex.
    \a. I bought [$_{DP}$ the book 
        [$_{CP}$ which$_{i}$ Suzuka hates $t_{i}$  ] 
    ].
    \b. I bought [$_{DP}$ the book 
        [$_{CP}$ which$_{i}$ Moa said
            [$_{CP}$ Suzuka hates $t_{i}$ ]
        ]
    ].

It can also raise recursively.

\ex. I bought [$_{DP}$ the book 
[$_{CP}$ which$_{i}$ Jimmy wonders
    [$_{CP}$ if Mao said 
        [$_{CP}$ Suzuka hates $t_{i}$ ]
    ]
]
].

However, the relative pronoun cannot raise from a clause that has a \emph{wh}-element in spec-CP,
such as the embedded indirect question in (4).

\ex. *I bought [$_{DP}$ the book 
    [$_{CP}$ which$_{i}$ Bonzo wonders who$_{j}$
        [$_{CP}$ $t_{j}$ thought 
            [$_{CP}$ Robert hates $t_{i}$ ]
        ]
    ]
].

We can correctly predict (4) to be ungrammatical if we assume that the relative pronoun undergoes
\emph{wh}-movement, in which case (4) would violate the \emph{Wh}-Island constraint.

Likewise, relative pronouns cannot raise from within the subject of the relative clause.

\ex. *I saw [$_{DP}$ the guy
    [$_{CP}$ who$_{i}$
        [$_{CP}$ that Stevie is dating $t_{i}$
            ]
        annoys Lindsey
        ]
    ].

In (5), the CP \emph{that Stevie is dating} is the subject of the relative clause. Raising the
relative pronoun from within this CP produces an ungrammatical sentence, suggesting that it
violates the Subject-Island constraint.

Relative clauses that raise a pronoun from a clause introduced by \emph{that} have dubious
acceptability.

\ex.
    \a. ??I bought [$_{DP}$ the book 
        [$_{CP}$ which$_{i}$ The Edge denied my claim
            [$_{CP}$ that Bono read $t_{i}$ today
            ]
        ]
    ].
    \b. ??I bought [$_{DP}$ the book 
        [$_{CP}$ which$_{i}$ Brian said
            [$_{CP}$ that $t_{i}$ offended Freddie
            ]
        ]
    ].

These sentences match the conditions for a \emph{That-trace} Island in \emph{wh}-movement.

Finally, raising a relative pronoun from the adjunct position is not consistently acceptable for
English speakers.

\ex. ??I bought [$_{DP}$ the book 
    [$_{CP}$ which$_{i}$ Stanton stole a newspaper near $t_{i}$ ]
].

It is possible that the sentence is less than acceptable because it violates the Adjunct Island
constraint.

Thus, relative pronouns appear to obey the same Island Constraints that \emph{wh}-phrases in
content questions observe. This indicates that relative pronouns raise by \emph{wh}-movement,
otherwise we would expect different constraints to affect relative pronoun raising than
those observed in the data.

\item \textbf{Reflection}

I learned a lot about how language structure can be described using a generative, tree-based model.
Obviously, that was the point of the class. But I think it's important to note that I came into
this class with a little familiarity with generative models of syntax, and a whole lot of skepticism
as to their explanatory power. This class repeatedly surprised me with not only how much our model
was able to explain English data, but how readily it could be applied to other languages. I enjoyed
the second half a lot, because it answered many questions I had as far as how verbs select arguments
and how embedded clauses are formed.

\item \textbf{BONUS}\\ Those legos he was playing with in class the other day, I think they
were called ``Nexo Knights''?
\item \textbf{BONUS}
\begin{enumerate}[label=\arabic*.]
    \setcounter{enumii}{5}
    \item Entailed
    \item Entailed. Even if ``untied'' has a nonliteral sense in (7a), the same 
    sense could be applied to (7b).
    \item Entailed
    \item Not implied. Typically when someone says ``alleged'' outside of a legalistic
    context, I understand they do not themselves believe the alleged claim, so I would not
    expect meeting an ``alleged communist'' presupposes meeting an actual communist.
    \item Entailed
    \item Not implied. At least, Gawker is not presupposed any more than Z Magazine is.
    \item Presupposed. If I reported that David agreed to eat breakfast, it could have
    the implication that David agrees to whatever circumstances I/anyone else decide,
    which is not entailed by agreeing to breakfast only at 8:00.
    \item Not implied
    \item The maxim of quantity.
    \item The maxim of relevance.
    \item The maxim of quality.
    \item The maxim of quantity.
\end{enumerate}

\end{enumerate}

\end{document}