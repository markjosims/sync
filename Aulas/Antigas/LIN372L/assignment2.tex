\documentclass[12pt]{article}

% MLA format
\usepackage[letterpaper]{geometry}
\usepackage{times}
\geometry{top=1.0in, bottom=1.0in, left=1.0in, right=1.0in}
\usepackage{fancyhdr}
\pagestyle{fancy}
\lhead{} 
\chead{} 
\rhead{Simmons \thepage} 
\lfoot{} 
\cfoot{} 
\rfoot{}
\renewcommand{\headrulewidth}{0pt} 
\renewcommand{\footrulewidth}{0pt} 

\usepackage{mdwlist}
\usepackage{enumitem}
\title{Homework 2}
\author{Mark Simmons}
\date{February 7, 2020}

% Multi-line glosses
\usepackage{chngcntr}
\usepackage{gb4e,cgloss4e}

\newcounter{glossnum}

\newcommand{\numgloss}{\refstepcounter{glossnum}\alph{glossnum}.\space}
\counterwithin{glossnum}{xnumi}
\renewcommand{\theglossnum}{\thexnumi\alph{glossnum}}

% Typing in IPA
\usepackage{tipa}


\begin{document}

\maketitle

\begin{enumerate}

% question 1
\item Part of Speech Identification
% sentence w/ pos-tags
\begin{exe}
\ex\label{ex1}
\gll
The large evil leathery alligator has complained rather loudly to his aging keeper about his extremely unattractive description.\\
DET ADJ ADJ ADJ N AUX V ADV ADV PREP DET ADJ N PREP PRO ADV ADJ N.\\
\end{exe}


% justification for each pos tag of type V, N, ADJ or ADV
\begin{itemize}

\item \emph{large} (Adj): follows a determiner and precedes another Adj.
\item \emph{evil} (Adj): follows an Adj and precedes another Adj.
\item \emph{leathery} (Adj): has a common Adj suffix \emph{-y} and precedes a N
\item \emph{alligator} (N): follows a string of adjectives and precedes an Aux.
\item \emph{complained} (V): follows an Aux and bears past/passive participle suffix \emph{-ed}
\item \emph{rather} (Adv): follows a V and precedes another Adv
\item \emph{loudly} (Adv): follows another Adv and bears adverbial suffix \emph{-ly}
\item \emph{aging} (Adj): follows a Det and has a present participle suffix \emph{-ing}
\item \emph{keeper} (N): follows an Adj and bears nomen agenti suffix \emph{-er}
\item \emph{extremely} (Adv): precedes an Adj and bears adverbial suffix \emph{-ly}
\item \emph{unattractive} (Adj): bears adjectivial negative prefix \emph{un-} and precedes a N.
\item \emph{description} (N): follows an Adj and bears nominal suffix \emph{-tion}

\end{itemize}

% question 2
\item Nootka

% sample data from book
\begin{exe}
  \ex\label{par2} % Arabic numbering
  \begin{xlist} % the next ones, alphabetic numbering  as usual
    \ex\label{par2a} \textipa{mamu:k-ma qu:Pas-Pi}\\
    working-\textsc{pres} man-\textsc{def}\\
    ``The man is working.''
    \ex\label{par2b} \textipa{qu:Pas-ma mamu:k-Pi}\\
    man-\textsc{pres} work-\textsc{def}\\
    ``The one working is a man.''
  \end{xlist}
\end{exe}

% questions
\begin{enumerate}
\item \emph{qu\textipa{:P}as} can be assumed to function as a noun in \ref{par2a}, given that it bears morphology indicating definiteness and is translated as a noun in English.

\item \emph{mamu\textipa{:}k} functions as a verb in \ref{par2a}, given that it bears tense morphology and is translated as a verb in English.

\item \emph{qu\textipa{:P}as} functions as a noun in \ref{par2b}, given that it bears tense morphology and is translated into English as the predicate ``is a man.''

\item \emph{mamu\textipa{:}k} functions as a noun in \ref{par2b}, given that it bears definite morphology and is translated into English as the noun phrase ``The working one.''

\item I assumed that the tense suffix \emph{-ma} and the definite suffix -\emph{i} associate with verbs and nouns respectively, on the basis of how similar morphology selects parts of speech in English. I also used the parts of speech of the semantic equivalents to the Nootka words in the English translation as a guide.

\item I hypothesize that some elements of Nootka morphology serve a ``double-duty'' of marking inflections of a particular word category as well as deriving members of that category. Hence, if we assume that the root \emph{qu\textipa{:P}as} is nominal, then the tense suffix \emph{-ma} can be assumed to derive a verb sense from it. My hypothesis predicts that, given a corpus of natural texts, a root such as \emph{qu\textipa{:P}as} should occur overwhelmingly more frequently with nominal morphology than with verbal. I also predict that certain inflections, such as e.g. a causative marker, will only be able to associate with verb roots and not with nouns.

\end{enumerate}

% question 3
\item Indonesian Syntax
\begin{enumerate}

\item Dictionary of lexemes\\
\begin{tabular}{ll}
Word   & Gloss                \\
apa    & INTRG                \\
rumah  & big                  \\
itu    & that                 \\
besar  & big                  \\
ya     & yes                  \\
tidak  & NEG (for adjectives) \\
anda   & 2s                   \\
mahal  & expensive            \\
mobil  & car                  \\
pak    & mister               \\
ini    & this                 \\
bukan  & NEG (for nouns)      \\
saya   & 1s                   \\
kepala & head                 \\
bagian & section             
\end{tabular}

\item 
\begin{itemize}
\item Declarative copular sentences are formed by apposition of the two noun phrases, or the noun phrase and adjective.

\item  Negative sentences have the same structure as declarative copular sentences, save for a particle that precedes the negated phrase. The particle \emph{bukan} negates adjectival predicates, and the particle \emph{tidak}, negates nominal predicates.

\item Polar interrogative copular sentences have the same structure as declarative copular sentences, save that they are introduced by the interrogative particle \emph{apa} at the beginning of the sentence.
\end{itemize}

\item Unlike Indonesian, SAE always requires an overt copula (this is not true for all English dialects, though). A sentence like *``That house big'' is ungrammatical. Also, English has the same negation for nominal predicates and adjectival predicates, e.g. ``I am not big'' and ``I am not a student.''

\item
\begin{exe}
\ex\label{par3} \begin{xlist}
\ex\label{par3a}
Apa ini isteri Stacey?
\ex\label{par3b}
Ya, ini isteri Stacey.
\ex\label{par3c}
Bukan, ini bukan isteri Stacey.
\end{xlist}\end{exe}

\begin{exe}
\ex\label{par4} \begin{xlist}
\ex\label{par4a}
Apa pesta jauh?
\ex\label{par4b}
Ya, pesta jauh.
\ex\label{par4c}
Tidak, pesta tidak jauh. Pesta pojok di.
\end{xlist} \end{exe}

\end{enumerate}

\end{enumerate}



\end{document}
